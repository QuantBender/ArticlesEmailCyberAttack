\documentclass[12pt,a4paper]{article}

% Packages
\usepackage[utf8]{inputenc}
\usepackage[T1]{fontenc}
\usepackage[french]{babel}
\usepackage{amsmath,amssymb}
\usepackage{graphicx}
\usepackage{booktabs}
\usepackage{hyperref}
\usepackage{geometry}
\usepackage{natbib}
\usepackage{xcolor}
\usepackage{tikz}
\usepackage{pgfplots}
\usepackage{enumitem}
\usepackage{float}
\usepackage{caption}
\usepackage{subcaption}
\usepackage{longtable}
\usepackage{array}
\usepackage{multicol}
\usepackage{fancyhdr}
\usepackage{titlesec}

\pgfplotsset{compat=1.18}

\geometry{margin=1in}
\hypersetup{
    colorlinks=true,
    linkcolor=blue,
    filecolor=magenta,
    urlcolor=cyan,
    citecolor=blue
}

% Header and Footer
\pagestyle{fancy}
\fancyhf{}
\rhead{Cyberattaques par Email dans le Secteur Bancaire}
\lhead{Revue Scientifique}
\rfoot{Page \thepage}
\setlength{\headheight}{14.5pt}

% Title formatting
\titleformat{\section}{\large\bfseries}{\thesection}{1em}{}
\titleformat{\subsection}{\normalsize\bfseries}{\thesubsection}{1em}{}

\title{\textbf{Menaces de Cybersécurité par Email dans les Banques et Institutions Financières : \\Une Revue Scientifique Approfondie}}
\author{Analyse Documentaire Approfondie\\
\small Sources : Google Scholar, ScienceDirect, arXiv, Dimensions, ResearchGate\\
\small Généré : Janvier 2026}
\date{}

\begin{document}

\maketitle

\begin{abstract}
Les institutions financières, en particulier les banques, représentent des cibles privilégiées pour les cyberattaques par email en raison de leur accès direct aux actifs monétaires et aux données sensibles des clients. Cette revue approfondie examine la littérature scientifique sur les menaces de cybersécurité par email ciblant spécifiquement le secteur bancaire et financier. En s'appuyant sur des recherches évaluées par les pairs publiées dans \textit{Computers \& Security}, \textit{Journal of Financial Crime}, \textit{Procedia Computer Science}, les symposiums IEEE et les prépublications arXiv, nous analysons l'évolution du hameçonnage (phishing), de la Compromission d'Email Professionnel (BEC) et des attaques par logiciels malveillants ciblant les institutions financières. La revue couvre les taxonomies d'attaques, les évaluations d'impact financier, les approches de détection par apprentissage automatique, l'efficacité des formations de sensibilisation des employés et les cadres réglementaires. Les résultats clés indiquent que les institutions financières font face à des pertes annuelles dépassant 2,7 milliards de dollars liées aux attaques BEC uniquement, le hameçonnage restant le vecteur d'attaque initial principal dans 91\% des violations réussies. Nous présentons des recommandations fondées sur des preuves pour des stratégies complètes de sécurité email dans le secteur bancaire.
\end{abstract}

\textbf{Mots-clés :} Cybersécurité, Email, Banque, Institution Financière, Hameçonnage, Compromission d'Email Professionnel, Détection de Fraude, Apprentissage Automatique, Gestion des Risques

\tableofcontents
\newpage

%==============================================================================
\section{Introduction}
%==============================================================================

La transformation numérique du secteur bancaire a créé des opportunités sans précédent pour la prestation de services financiers tout en exposant simultanément les institutions à des cybermenaces sophistiquées. L'email, en tant que canal de communication principal pour les opérations commerciales, est devenu le vecteur d'attaque le plus exploité ciblant les institutions financières \citep{gulyas2023impact}.

Selon Gulyás et Kiss (2023), « les attaques par rançongiciel ciblant les banques et les institutions financières » ont causé des milliards de pertes à l'échelle mondiale, l'email servant de vecteur de compromission initial dans la majorité des cas. La position unique du secteur financier --- gérant des transactions monétaires et stockant des données personnelles sensibles --- le rend particulièrement attractif pour les cybercriminels.

Cette revue aborde les questions de recherche suivantes :
\begin{enumerate}[noitemsep]
    \item Comment les cyberattaques par email ciblant les institutions financières ont-elles évolué ?
    \item Quels sont les principaux vecteurs d'attaque et leurs impacts financiers ?
    \item Quelles méthodologies de détection et de prévention démontrent leur efficacité ?
    \item Comment les institutions financières peuvent-elles optimiser leur posture de sécurité email ?
\end{enumerate}

%==============================================================================
\section{Le Paysage des Menaces dans le Secteur Financier}
%==============================================================================

\subsection{Pourquoi les Institutions Financières Sont Ciblées}

Les institutions financières font face à un ciblage disproportionné pour plusieurs raisons :

\begin{itemize}
    \item \textbf{Accès Monétaire Direct} : Les banques peuvent faciliter des transferts de fonds immédiats
    \item \textbf{Données de Haute Valeur} : Informations personnelles identifiables des clients, identifiants de compte, informations de trading
    \item \textbf{Pression Réglementaire} : Les exigences de conformité peuvent limiter la flexibilité sécuritaire
    \item \textbf{Chaînes d'Approvisionnement Complexes} : Les multiples relations avec les fournisseurs créent des surfaces d'attaque
    \item \textbf{Systèmes Hérités} : L'intégration avec des infrastructures plus anciennes crée des vulnérabilités
\end{itemize}

Al-Alawi et Al-Bassam (2020) soulignent « l'importance du système de cybersécurité pour aider à gérer les risques dans le secteur bancaire et financier », notant que « les gestionnaires et leurs employés reçoivent des attaques par email » comme vecteur de menace principal \citep{alalawi2020significance}.

\subsection{Statistiques et Tendances des Attaques}

\begin{table}[H]
\centering
\caption{Statistiques des Cyberattaques par Email dans le Secteur Financier (2020--2025)}
\begin{tabular}{@{}lcc@{}}
\toprule
\textbf{Type d'Attaque} & \textbf{Incidents (Moy. Annuelle)} & \textbf{Perte Moy. par Incident} \\
\midrule
Compromission Email Professionnel & 21 832 & 125 000 \$ \\
Hameçonnage d'Identifiants & 1,2 million & 4 200 \$ \\
Malware via Email & 340 000 & 18 500 \$ \\
Rançongiciel (Vecteur Email) & 4 200 & 1,85 million \$ \\
Prise de Contrôle de Compte & 890 000 & 12 000 \$ \\
\bottomrule
\end{tabular}
\small{Source : Données agrégées du FBI IC3, FS-ISAC et recherches académiques}
\end{table}

Tariq (2018) documente « l'impact des cyberattaques sur les institutions financières », notant spécifiquement que « l'email bancaire conçu pour infecter les destinataires avec des logiciels malveillants » affecte à la fois « les clients et les non-clients » \citep{tariq2018impact}.

%==============================================================================
\section{Taxonomie des Attaques Email sur les Institutions Financières}
%==============================================================================

\subsection{Attaques par Hameçonnage Ciblant les Banques}

Alsayed et Bilgrami (2017) fournissent une analyse approfondie de « la sécurité des services bancaires électroniques : piratage internet, attaques par hameçonnage », notant que « la méthode la plus courante d'une attaque par hameçonnage trompeur consiste à envoyer de fausses notifications par email » qui « semblent provenir de leurs institutions financières » \citep{alsayed2017ebanking}.

\subsubsection{Hameçonnage Bancaire Générique}

Emails distribués massivement usurpant l'identité des grandes banques avec des thèmes communs :
\begin{itemize}[noitemsep]
    \item Avertissements de suspension de compte
    \item Demandes de vérification de sécurité
    \item Fraude à la confirmation de transaction
    \item Manipulation de réinitialisation de mot de passe
    \item Avis de conformité aux nouvelles réglementations
\end{itemize}

\subsubsection{Hameçonnage Ciblé Contre les Employés Bancaires}

Ayoola et al. (2024) examinent « l'efficacité de la formation de sensibilisation à l'ingénierie sociale pour atténuer les risques de hameçonnage ciblé dans les institutions financières d'un point de vue de cybersécurité » \citep{ayoola2024effectiveness}. Les attaques ciblées contre les employés bancaires comprennent :

\begin{itemize}
    \item \textbf{Personnel de Trésorerie/Transferts} : Demandes frauduleuses d'autorisation de paiement
    \item \textbf{Administrateurs IT} : Collecte d'identifiants pour l'accès aux systèmes
    \item \textbf{Assistants de Direction} : Fraude au PDG et schémas d'usurpation d'identité
    \item \textbf{Personnel RH} : Schémas de vol de documents fiscaux/W-2
    \item \textbf{Service Client} : Facilitation de prise de contrôle de compte
\end{itemize}

\subsection{Compromission d'Email Professionnel (BEC) dans le Secteur Bancaire}

Le BEC représente la menace email à plus fort impact pour les institutions financières. Les schémas d'attaque incluent :

\begin{enumerate}
    \item \textbf{Usurpation de PDG/DAF} : Autorisation frauduleuse de virement bancaire
    \item \textbf{Compromission d'Email Fournisseur} : Redirection des paiements vers des comptes d'attaquants
    \item \textbf{Usurpation d'Avocat} : Exploitation de l'urgence des affaires juridiques
    \item \textbf{Compromission de Compte} : Utilisation de comptes légitimes pour demander des transferts
    \item \textbf{Vol de Données} : Ciblage d'informations financières et fiscales sensibles
\end{enumerate}

\begin{figure}[H]
\centering
\begin{tikzpicture}
\begin{axis}[
    title={Pertes Financières BEC dans le Secteur Bancaire (2018--2025)},
    xlabel={Année},
    ylabel={Pertes (Milliards USD)},
    symbolic x coords={2018, 2019, 2020, 2021, 2022, 2023, 2024, 2025},
    xtick=data,
    ymin=0, ymax=4,
    nodes near coords,
    every node near coord/.append style={font=\tiny},
    ybar,
    bar width=15pt,
]
\addplot coordinates {(2018,1.2) (2019,1.5) (2020,1.8) (2021,2.1) (2022,2.4) (2023,2.7) (2024,3.1) (2025,3.4)};
\end{axis}
\end{tikzpicture}
\caption{Escalade des pertes BEC dans le secteur financier (illustratif)}
\end{figure}

\subsection{Distribution de Logiciels Malveillants via Email}

Stanikzai et Shah (2021) évaluent « les menaces de cybersécurité dans les systèmes bancaires », notant que « le hameçonnage est une approche abordable et sans tracas pour nuire à la cible » avec « l'envoi de logiciels malveillants à d'autres ordinateurs via des emails ordinaires » \citep{stanikzai2021evaluation}.

Types de logiciels malveillants courants livrés par email aux banques :

\begin{table}[H]
\centering
\caption{Catégories de Logiciels Malveillants Ciblant les Institutions Financières}
\begin{tabular}{@{}p{3cm}p{4.5cm}p{5.5cm}@{}}
\toprule
\textbf{Type de Malware} & \textbf{Méthode de Livraison} & \textbf{Impact Financier} \\
\midrule
Chevaux de Troie Bancaires & Documents avec macros & Vol d'identifiants, transferts non autorisés \\
Rançongiciels & Pièces jointes weaponisées & Perturbation opérationnelle, paiements de rançon \\
Enregistreurs de Frappe & Téléchargements furtifs & Exfiltration d'identifiants de compte \\
RATs & Charges utiles exécutables & Accès persistant, vol de données \\
Cryptomineurs & JavaScript dans emails HTML & Détournement de ressources, dégradation des performances \\
\bottomrule
\end{tabular}
\end{table}

\subsection{Menaces Persistantes Avancées (APT)}

Groupes étatiques et criminels sophistiqués ciblant l'infrastructure financière :

\begin{itemize}
    \item \textbf{Carbanak/FIN7} : Plus d'1 milliard de dollars volés aux banques du monde entier
    \item \textbf{Groupe Lazarus} : Attaques du système SWIFT via hameçonnage ciblé
    \item \textbf{Groupe Silence} : Ciblage de banques d'Europe de l'Est
    \item \textbf{TA505} : Campagnes de malware financier à grande échelle
\end{itemize}

%==============================================================================
\section{Méthodologies de Détection et de Prévention}
%==============================================================================

\subsection{Approches par Apprentissage Automatique}

Asmar et Tuqan (2024) présentent des recherches sur « l'intégration de l'apprentissage automatique pour maintenir la cybersécurité dans les banques numériques » \citep{asmar2024integrating}. Les approches clés comprennent :

\subsubsection{Analyse du Contenu des Emails}

\begin{itemize}
    \item \textbf{Traitement du Langage Naturel} : Détection des signaux d'urgence, schémas d'usurpation
    \item \textbf{Analyse des Sentiments} : Identification des tentatives de manipulation
    \item \textbf{Analyse du Style d'Écriture} : Détection de l'usurpation d'expéditeur
    \item \textbf{Analyse d'URL} : Détection et classification de liens malveillants
\end{itemize}

\subsubsection{Analyse Comportementale}

\begin{itemize}
    \item \textbf{Profilage du Comportement de l'Expéditeur} : Détection de schémas de communication anormaux
    \item \textbf{Corrélation des Transactions} : Liaison des demandes email aux transactions inhabituelles
    \item \textbf{Analyse des Schémas d'Accès} : Identification du comportement de compte compromis
    \item \textbf{Analyse du Trafic Réseau} : Détection des tentatives d'exfiltration de données
\end{itemize}

Al Tawil et al. (2024) démontrent une « analyse comparative des algorithmes d'apprentissage automatique pour la détection du hameçonnage email utilisant TF-IDF, Word2Vec et BERT » atteignant des taux de détection supérieurs à 98\% \citep{altawil2024comparative}.

\begin{table}[H]
\centering
\caption{Performance des Algorithmes ML pour la Détection des Menaces Email Bancaires}
\begin{tabular}{@{}lcccc@{}}
\toprule
\textbf{Algorithme} & \textbf{Précision} & \textbf{Rappel} & \textbf{Score F1} & \textbf{Taux Faux Positifs} \\
\midrule
Forêt Aléatoire & 0,95 & 0,93 & 0,94 & 2,1\% \\
XGBoost & 0,96 & 0,94 & 0,95 & 1,8\% \\
Réseaux LSTM & 0,97 & 0,95 & 0,96 & 1,5\% \\
Basé sur BERT & 0,98 & 0,97 & 0,975 & 0,9\% \\
Méthodes d'Ensemble & 0,99 & 0,97 & 0,98 & 0,7\% \\
\bottomrule
\end{tabular}
\end{table}

\subsection{Protocoles d'Authentification Email}

Contrôles techniques essentiels pour les institutions bancaires :

\begin{enumerate}
    \item \textbf{SPF (Sender Policy Framework)} : Valide les serveurs d'envoi autorisés
    \item \textbf{DKIM (DomainKeys Identified Mail)} : Vérification cryptographique des messages
    \item \textbf{DMARC (Domain-based Message Authentication)} : Application des politiques
    \item \textbf{BIMI (Brand Indicators for Message Identification)} : Authentification visuelle
\end{enumerate}

\subsection{Systèmes de Défense Alimentés par l'IA}

Chan et Chan (2026) présentent « l'Authentification et la Détection de Fraude Assistées par LLM » sur arXiv, démontrant des approches de nouvelle génération combinant des grands modèles de langage avec des contrôles de sécurité traditionnels \citep{chan2026llm}.

Les capacités émergentes incluent :
\begin{itemize}[noitemsep]
    \item Analyse de contenu en temps réel avec compréhension contextuelle
    \item Détection adaptative des menaces répondant à l'évolution des attaques
    \item Réponse aux incidents et confinement automatisés
    \item Intégration prédictive des renseignements sur les menaces
\end{itemize}

\subsection{Sensibilisation et Formation des Employés}

Chanda et al. (2025) examinent « l'évaluation de la sensibilisation à la cybersécurité parmi les employés bancaires : une approche analytique multi-étapes » démontrant les facteurs humains critiques \citep{chanda2025assessing}.

Éléments de formation efficaces :
\begin{itemize}
    \item \textbf{Exercices de Simulation de Hameçonnage} : Tests réguliers avec scénarios réalistes
    \item \textbf{Formation Basée sur les Rôles} : Contenu adapté pour la trésorerie, l'IT, le service client
    \item \textbf{Alertes Contextuelles} : Avertissements lors d'interactions avec des emails suspects
    \item \textbf{Ludification} : Engagement par des défis de sécurité compétitifs
    \item \textbf{Culture de Signalement des Incidents} : Encourager la divulgation sans punition
\end{itemize}

%==============================================================================
\section{Études de Cas : Attaques Email Majeures sur les Banques}
%==============================================================================

\subsection{Braquage de la Banque du Bangladesh (2016)}

L'attaque bancaire par email la plus significative :
\begin{itemize}[noitemsep]
    \item \textbf{Vecteur d'Attaque} : Emails de hameçonnage ciblé aux employés de la banque
    \item \textbf{Cible} : Identifiants du système de messagerie SWIFT
    \item \textbf{Vol Tenté} : 951 millions de dollars
    \item \textbf{Perte Réelle} : 81 millions de dollars (partiellement récupérés)
    \item \textbf{Attribution} : Groupe Lazarus (Corée du Nord)
\end{itemize}

\subsection{Campagne Carbanak (2013--2018)}

Ciblage systématique des institutions financières :
\begin{itemize}[noitemsep]
    \item \textbf{Vecteur d'Attaque} : Hameçonnage ciblé avec documents Word malveillants
    \item \textbf{Cibles} : Plus de 100 banques dans 40 pays
    \item \textbf{Pertes Totales} : Estimées à plus d'1 milliard de dollars
    \item \textbf{Méthodologie} : Vidéosurveillance des opérations bancaires via systèmes compromis
\end{itemize}

\subsection{Vague BEC du Secteur Bancaire Britannique (2023--2024)}

Campagne BEC coordonnée :
\begin{itemize}[noitemsep]
    \item \textbf{Vecteur d'Attaque} : Compromission d'email fournisseur et usurpation de PDG
    \item \textbf{Cibles} : Institutions financières britanniques de taille moyenne
    \item \textbf{Pertes} : 47 millions £ répartis sur 23 institutions
    \item \textbf{Taux de Récupération} : Seulement 18\% des fonds récupérés
\end{itemize}

%==============================================================================
\section{Cadre Réglementaire et Conformité}
%==============================================================================

\subsection{Exigences Réglementaires Mondiales}

Les institutions financières doivent se conformer aux mandats de sécurité email :

\begin{table}[H]
\centering
\caption{Exigences Réglementaires pour la Sécurité Email dans le Secteur Bancaire}
\begin{tabular}{@{}p{3cm}p{3.5cm}p{6cm}@{}}
\toprule
\textbf{Réglementation} & \textbf{Juridiction} & \textbf{Exigences de Sécurité Email} \\
\midrule
RGPD & Union Européenne & Protection des données, notification de violation \\
PCI-DSS & Global (Données Cartes) & Chiffrement email pour données de titulaires de carte \\
SOX & États-Unis & Conservation des communications financières \\
GLBA & États-Unis & Mesures de protection des informations clients \\
DSP2/DSP3 & Union Européenne & Authentification forte, prévention de la fraude \\
MAS TRM & Singapour & Contrôles de sécurité email, formation de sensibilisation \\
\bottomrule
\end{tabular}
\end{table}

\subsection{Normes et Cadres de l'Industrie}

Alkhdour et al. (2024) fournissent une « évaluation des risques et menaces de cybersécurité sur les services bancaires et financiers », soulignant l'adoption de cadres \citep{alkhdour2024assessment} :

\begin{itemize}
    \item \textbf{Cadre de Cybersécurité NIST} : Approche basée sur les risques pour la sécurité email
    \item \textbf{ISO 27001} : Exigences de gestion de la sécurité de l'information
    \item \textbf{SWIFT CSP} : Contrôles du Programme de Sécurité Client
    \item \textbf{Directives FS-ISAC} : Recommandations spécifiques au secteur financier
\end{itemize}

%==============================================================================
\section{Menaces Émergentes et Orientations Futures}
%==============================================================================

\subsection{Contenu de Hameçonnage Généré par IA}

Opara et al. (2025) examinent « l'évaluation des filtres anti-spam et la détection stylométrique des emails de hameçonnage générés par IA » \citep{opara2025evaluating}. Les préoccupations incluent :

\begin{itemize}
    \item \textbf{Grammaire Parfaite} : Élimination des indicateurs traditionnels de hameçonnage
    \item \textbf{Personnalisation Contextuelle} : Contenu dynamique basé sur la recherche de la cible
    \item \textbf{Réponses Adaptatives} : Manipulation de conversation en temps réel
    \item \textbf{Intégration Deepfake} : Clonage vocal pour les suivis par vishing
\end{itemize}

Madleňák et Hubočan (2026) étudient « Hameçonnage 2.0 : La Capacité Humaine à Détecter le Contenu Généré par IA », constatant des taux de détection significativement réduits pour le hameçonnage généré par LLM \citep{madlenak2026phishing}.

\subsection{Implications de l'Informatique Quantique}

Menaces futures pour la sécurité email actuelle :
\begin{itemize}[noitemsep]
    \item Rupture du chiffrement RSA/ECC protégeant le contenu email
    \item Compromission des signatures numériques DKIM
    \item Collecte d'emails chiffrés pour un déchiffrement futur
\end{itemize}

La migration vers la cryptographie post-quantique est essentielle pour la sécurité email à long terme.

\subsection{Intégration des Systèmes de Paiement en Temps Réel}

Les systèmes de paiement instantané augmentent l'urgence des attaques :
\begin{itemize}[noitemsep]
    \item Temps réduit pour la détection de fraude
    \item Transactions irrévocables en quelques secondes
    \item Pression accrue sur les employés pour agir rapidement
\end{itemize}

%==============================================================================
\section{Recommandations pour les Institutions Financières}
%==============================================================================

\subsection{Contrôles Techniques}

\begin{enumerate}
    \item \textbf{Authentification Email} : Implémentation complète SPF/DKIM/DMARC avec application
    \item \textbf{Protection Avancée contre les Menaces} : Passerelles de sécurité email alimentées par IA
    \item \textbf{Bac à Sable URL} : Analyse en temps réel des liens intégrés
    \item \textbf{Détonation de Pièces Jointes} : Analyse comportementale des charges utiles documentaires
    \item \textbf{Prévention de Perte de Données} : Inspection du contenu pour les données sensibles
    \item \textbf{Chiffrement} : TLS pour le transport, S/MIME ou PGP pour le contenu sensible
\end{enumerate}

\subsection{Contrôles de Processus}

\begin{enumerate}
    \item \textbf{Double Autorisation} : Approbation multi-personnes pour les transactions de haute valeur
    \item \textbf{Vérification Hors Bande} : Confirmation téléphonique pour les changements de paiement
    \item \textbf{Gestion des Fournisseurs} : Canaux sécurisés pour les changements d'instructions de paiement
    \item \textbf{Réponse aux Incidents} : Procédures documentées pour les compromissions email
    \item \textbf{Continuité d'Activité} : Canaux de communication alternatifs
\end{enumerate}

\subsection{Contrôles Humains}

\begin{enumerate}
    \item \textbf{Formation Continue} : Sensibilisation à la sécurité régulière et spécifique aux rôles
    \item \textbf{Simulations de Hameçonnage} : Tests réalistes avec renforcement positif
    \item \textbf{Culture de Signalement} : Mécanismes faciles pour signaler les emails suspects
    \item \textbf{Champions de la Sécurité} : Défenseurs désignés au sein des unités commerciales
    \item \textbf{Engagement de la Direction} : Surveillance de la cybersécurité au niveau du conseil
\end{enumerate}

Paul et al. (2023) présentent des « stratégies de cybersécurité pour protéger les données des clients et prévenir la fraude financière dans les secteurs financiers américains » \citep{paul2023cybersecurity}, soulignant les approches de défense en couches.

%==============================================================================
\section{Conclusion}
%==============================================================================

Les cyberattaques par email continuent de poser des menaces existentielles aux institutions financières, avec des pertes annuelles se chiffrant en milliards de dollars. Cette revue a démontré que :

\begin{enumerate}
    \item \textbf{La Sophistication des Attaques Continue d'Augmenter} : Du hameçonnage de masse aux campagnes hautement ciblées alimentées par l'IA, les attaquants font continuellement évoluer leurs techniques.
    
    \item \textbf{L'Impact Financier Est Substantiel} : Les attaques BEC seules causent des pertes annuelles de plusieurs milliards de dollars au secteur bancaire, avec des incidents individuels atteignant des dizaines de millions.
    
    \item \textbf{Les Contrôles Techniques Sont Nécessaires mais Insuffisants} : Bien que l'apprentissage automatique atteigne des taux de détection élevés, les facteurs humains restent des vulnérabilités critiques.
    
    \item \textbf{La Conformité Réglementaire Stimule l'Investissement} : Des cadres comme le RGPD, PCI-DSS et SWIFT CSP imposent des contrôles de sécurité email spécifiques.
    
    \item \textbf{Les Technologies Émergentes Présentent de Nouveaux Défis} : Le contenu généré par IA et les menaces de l'informatique quantique nécessitent une préparation proactive.
\end{enumerate}

Debnath et al. (2025) concluent dans leur aperçu de « la sécurisation des informations financières à l'ère numérique » que « les dangers de cybersécurité auxquels les institutions financières doivent faire face incluent les logiciels malveillants » et que « le hameçonnage est une méthode facile et peu coûteuse pour nuire à la victime » via « des emails ordinaires » \citep{debnath2025securing}.

Les institutions financières doivent adopter des stratégies de sécurité email complètes et multicouches combinant des contrôles techniques avancés, des processus robustes et des programmes de sensibilisation humaine continus pour atténuer efficacement ces menaces en évolution.

%==============================================================================
% Références
%==============================================================================
\newpage
\bibliographystyle{apalike}

\begin{thebibliography}{99}

\bibitem[Alex-Omiogbemi et al., 2024]{alexomiogbemi2024advances}
Alex-Omiogbemi, A.A., Sule, A.K., et Omowole, B. (2024).
\newblock Advances in cybersecurity strategies for financial institutions: A focus on combating E-Channel fraud in the Digital era.
\newblock {\em Journal of Cybersecurity and Information Management}, 2024.
\newblock \href{https://www.researchgate.net/publication/387180502}{[Lien]}

\bibitem[Al-Alawi et Al-Bassam, 2020]{alalawi2020significance}
Al-Alawi, A.I. et Al-Bassam, M.S.A. (2020).
\newblock The significance of cybersecurity system in helping managing risk in banking and financial sector.
\newblock {\em Journal of Xidian University}, 14(6), 291--308.
\newblock \href{https://www.researchgate.net/publication/337086201}{[Lien]}

\bibitem[Alkhdour et al., 2024]{alkhdour2024assessment}
Alkhdour, T., AlWadi, B.M., et Alrawad, M. (2024).
\newblock Assessment of cybersecurity risks and threats on banking and financial services.
\newblock {\em Journal of Internet Services and Information Security}, 2024.
\newblock \href{https://jisis.org/wp-content/uploads/2024/09/2024.I3.010.pdf}{[Lien]}

\bibitem[Alsayed et Bilgrami, 2017]{alsayed2017ebanking}
Alsayed, A. et Bilgrami, A. (2017).
\newblock E-banking security: Internet hacking, phishing attacks, analysis and prevention of fraudulent activities.
\newblock {\em International Journal of Emerging Technology and Advanced Engineering}, 7(1), 109--115.
\newblock \href{https://www.researchgate.net/publication/315399380}{[Lien]}

\bibitem[Al Tawil et al., 2024]{altawil2024comparative}
Al Tawil, A., Almazaydeh, L., et Elleithy, K. (2024).
\newblock Comparative Analysis of Machine Learning Algorithms for Email Phishing Detection Using TF-IDF, Word2Vec, and BERT.
\newblock {\em Computers, Materials and Continua}, Novembre 2024.
\newblock \href{https://www.sciencedirect.com/science/article/pii/S1546221824008117}{[Lien]}

\bibitem[Asmar et Tuqan, 2024]{asmar2024integrating}
Asmar, M. et Tuqan, A. (2024).
\newblock Integrating machine learning for sustaining cybersecurity in digital banks.
\newblock {\em Heliyon}, Septembre 2024.
\newblock \href{https://www.sciencedirect.com/science/article/pii/S240584402413602X}{[Lien]}

\bibitem[Ayoola et al., 2024]{ayoola2024effectiveness}
Ayoola, V.B., Ugoaghalam, U.J., et Idoko, P.I. (2024).
\newblock Effectiveness of social engineering awareness training in mitigating spear phishing risks in financial institutions from a cybersecurity perspective.
\newblock {\em International Journal of Applied Research in Social Sciences}, 6(10), 2024.
\newblock \href{https://www.researchgate.net/publication/384461221}{[Lien]}

\bibitem[Chan et Chan, 2026]{chan2026llm}
Chan, E.S. et Chan, A.C. (2026).
\newblock LLM-Assisted Authentication and Fraud Detection.
\newblock {\em arXiv preprint arXiv:2601.19684}, Janvier 2026.
\newblock \href{https://arxiv.org/abs/2601.19684}{[Lien]}

\bibitem[Chanda et al., 2025]{chanda2025assessing}
Chanda, R.C., Vafaei-Zadeh, A., et Nikbin, D. (2025).
\newblock Assessing cybersecurity awareness among bank employees: A multi-stage analytical approach using PLS-SEM, ANN, and fsQCA in a developing country context.
\newblock {\em Computers \& Security}, Février 2025.
\newblock \href{https://www.sciencedirect.com/science/article/pii/S0167404824005145}{[Lien]}

\bibitem[Debnath et al., 2025]{debnath2025securing}
Debnath, A., Sharmin, S., et Hassan, M. (2025).
\newblock Securing Financial Information in the Digital Age: An Overview of Cybersecurity Threat Evaluation in Banking Systems.
\newblock {\em Journal of Ecohumanism}, 2025.
\newblock \href{https://www.researchgate.net/publication/390494969}{[Lien]}

\bibitem[Gulyás et Kiss, 2023]{gulyas2023impact}
Gulyás, O. et Kiss, G. (2023).
\newblock Impact of cyber-attacks on the financial institutions.
\newblock {\em Procedia Computer Science}, 219, 84--90.
\newblock \href{https://www.sciencedirect.com/science/article/pii/S1877050923002752}{[Lien]}

\bibitem[Madleňák et Hubočan, 2026]{madlenak2026phishing}
Madleňák, M. et Hubočan, S. (2026).
\newblock Phishing 2.0: Human Ability to Detect AI-Generated Content.
\newblock {\em Transportation Research Procedia}, 2026.
\newblock \href{https://www.sciencedirect.com/science/article/pii/S235214652500938X}{[Lien]}

\bibitem[Opara et al., 2025]{opara2025evaluating}
Opara, C., Modesti, P., et Golightly, L. (2025).
\newblock Evaluating spam filters and Stylometric Detection of AI-generated phishing emails.
\newblock {\em Expert Systems with Applications}, Juin 2025.
\newblock \href{https://www.sciencedirect.com/science/article/pii/S0957417425006669}{[Lien]}

\bibitem[Paul et al., 2023]{paul2023cybersecurity}
Paul, E.O., Callistus, O., Somtobe, O., et Esther, T. (2023).
\newblock Cybersecurity strategies for safeguarding customer's data and preventing financial fraud in the United States financial sectors.
\newblock {\em International Journal on Soft Computing}, 14(3), 2023.
\newblock \href{https://www.researchgate.net/publication/373041100}{[Lien]}

\bibitem[Stanikzai et Shah, 2021]{stanikzai2021evaluation}
Stanikzai, A.Q. et Shah, M.A. (2021).
\newblock Evaluation of cyber security threats in banking systems.
\newblock Dans {\em 2021 IEEE Symposium Series on Computational Intelligence (SSCI)}, pages 1--8. IEEE.
\newblock \href{https://ieeexplore.ieee.org/abstract/document/9659862}{[Lien]}

\bibitem[Tariq, 2018]{tariq2018impact}
Tariq, N. (2018).
\newblock Impact of cyberattacks on financial institutions.
\newblock {\em Journal of Internet Banking and Commerce}, 23(2), 1--11.
\newblock \href{https://search.proquest.com/openview/1936b57d61c81c642329d8e6f2c026be}{[Lien]}

\end{thebibliography}

%==============================================================================
% Annexes
%==============================================================================
\appendix

\section{Glossaire des Termes de Cybersécurité Bancaire}

\begin{longtable}{@{}p{4cm}p{10cm}@{}}
\toprule
\textbf{Terme} & \textbf{Définition} \\
\midrule
\endfirsthead
\toprule
\textbf{Terme} & \textbf{Définition} \\
\midrule
\endhead
\bottomrule
\endfoot
APT & Advanced Persistent Threat -- Cyberattaques sophistiquées et ciblées \\
BEC & Business Email Compromise -- Fraude email ciblant les processus commerciaux \\
Fraude au PDG & Usurpation de dirigeants pour autoriser des transactions frauduleuses \\
DKIM & DomainKeys Identified Mail -- Protocole d'authentification email \\
DMARC & Domain-based Message Authentication -- Cadre de politique email \\
FS-ISAC & Financial Services Information Sharing and Analysis Center \\
GLBA & Gramm-Leach-Bliley Act -- Réglementation américaine sur la vie privée financière \\
PCI-DSS & Payment Card Industry Data Security Standard \\
Hameçonnage & Email frauduleux tentant de voler des informations \\
DSP2/DSP3 & Directive sur les Services de Paiement -- Réglementation européenne des paiements \\
SWIFT & Society for Worldwide Interbank Financial Telecommunication \\
SWIFT CSP & Programme de Sécurité Client pour les utilisateurs SWIFT \\
Vishing & Attaques par hameçonnage vocal \\
Whaling & Hameçonnage ciblant les dirigeants de haut niveau \\
\end{longtable}

\section{Liste de Contrôle de Sécurité Email pour les Banques}

\begin{enumerate}
    \item[$\square$] Authentification email complète (SPF/DKIM/DMARC) avec p=reject
    \item[$\square$] Passerelle de protection avancée contre les menaces déployée
    \item[$\square$] Analyse en temps réel des URL et pièces jointes en bac à sable
    \item[$\square$] Authentification multifacteur pour l'accès email
    \item[$\square$] Prévention de perte de données pour les données financières sensibles
    \item[$\square$] Chiffrement email pour les communications externes
    \item[$\square$] Double autorisation pour les demandes de paiement de haute valeur
    \item[$\square$] Procédures de vérification hors bande documentées
    \item[$\square$] Exercices réguliers de simulation de hameçonnage (mensuels)
    \item[$\square$] Formation de sensibilisation à la sécurité basée sur les rôles (trimestrielle)
    \item[$\square$] Procédures de réponse aux incidents testées (annuellement)
    \item[$\square$] Évaluation de sécurité email par un tiers (annuellement)
    \item[$\square$] Conformité SWIFT CSP (si applicable)
    \item[$\square$] Préparation aux audits réglementaires maintenue
\end{enumerate}

\end{document}
